\documentclass{tufte-handout}

%\title{An Example of the Usage of the Tufte-Handout Style\thanks{Inspired by Edward~R. Tufte!}}
\title{Two-Class Classification Task\thanks{CS-GY-6923\newline Course Instructor: Gang Li;\newline TA: Xu Zhao}}
%\author[The Tufte-LaTeX Developers]{The Tufte-\LaTeX\ Developers}
\author[Yun Yan]{Yun Yan (\href{mailto:yy1533@nyu.edu}{yy1533@nyu.edu})}
\date{Oct 20, 2015} % without \date command, current date is supplied

%\geometry{showframe} % display margins for debugging page layout

\usepackage{graphicx} % allow embedded images
  \setkeys{Gin}{width=\linewidth,totalheight=\textheight,keepaspectratio}
  \graphicspath{{/Users/yunyan/Yun_Codes/Git_Public_Pres_and_Blog/ML_course_Project/fig/}} % set of paths to search for images
\usepackage{amsmath}  % extended mathematics
\usepackage{booktabs} % book-quality tables
\usepackage{units}    % non-stacked fractions and better unit spacing
\usepackage{multicol} % multiple column layout facilities
\usepackage{lipsum}   % filler text
\usepackage{fancyvrb} % extended verbatim environments
  \fvset{fontsize=\normalsize}% default font size for fancy-verbatim environments

% Standardize command font styles and environments
\newcommand{\doccmd}[1]{\texttt{\textbackslash#1}}% command name -- adds backslash automatically
\newcommand{\docopt}[1]{\ensuremath{\langle}\textrm{\textit{#1}}\ensuremath{\rangle}}% optional command argument
\newcommand{\docarg}[1]{\textrm{\textit{#1}}}% (required) command argument
\newcommand{\docenv}[1]{\textsf{#1}}% environment name
\newcommand{\docpkg}[1]{\texttt{#1}}% package name
\newcommand{\doccls}[1]{\texttt{#1}}% document class name
\newcommand{\docclsopt}[1]{\texttt{#1}}% document class option name
\newenvironment{docspec}{\begin{quote}\noindent}{\end{quote}}% command specification environment

\usepackage{amsmath}
\usepackage{cancel}
\usepackage{amsthm}
\usepackage{bm}
\usepackage{amssymb}
\usepackage{algorithm}
\usepackage{algpseudocode}
\newcommand{\bigO}[1]{\ensuremath{\mathcal{O}\left(#1\right)}}
\newcommand{\bigT}[1]{\ensuremath{\Theta\left(#1\right)}}
\newcommand{\inv}{^{\raisebox{.2ex}{$\scriptscriptstyle-1$}}}

\usepackage{fancyhdr}
\usepackage{lastpage}
\fancyhf{}
\rhead{Page \thepage \hspace{1pt} of \pageref{LastPage}}
%\makeatletter
%\renewcommand{\@oddfoot}{\hfil \thepage/\pageref{LastPage} \hfil}
%\makeatother

\geometry{
	letterpaper,
	left = 0.5in, 
	marginparsep=2pc,
	marginparwidth = 17pc
%	marginparwidth=20pc
	}
%\geometry{
%  letterpaper, 
%  left=1in, % left margin
%  textwidth=25pc, % main text block
%  marginparsep=2pc, % gutter between main text block and margin notes
%  marginparwidth=12pc % width of margin notes
%}
%\usepackage[usenames, dvipsnames]{color}
\usepackage{xcolor}
\usepackage{color}
\setcounter{secnumdepth}{3}
\titleformat{\section}%
  {\normalfont\Large\itshape\color{Orange}}% format applied to label+text
  {\llap{\colorbox{Orange}{\parbox{1.5cm}{\hfill\color{white}\thesection}}}}% label
  {1em}% horizontal separation between label and title body
  {}% before the title body
  []% after the title body
\titleformat{\subsection}%
  {\normalfont\itshape\color{BurntOrange}}% format applied to label+text
  {\llap{\colorbox{BurntOrange}{\color{white}\thesubsection}}}% label
  {1em}% horizontal separation between label and title body
  {}% before the title body
  []% after the title body

\usepackage{array}
\newcommand{\PreserveBackslash}[1]{\let\temp=\\#1\let\\=\temp}
\newcolumntype{C}[1]{>{\PreserveBackslash\centering}p{#1}}
\newcolumntype{R}[1]{>{\PreserveBackslash\raggedleft}p{#1}}
\newcolumntype{L}[1]{>{\PreserveBackslash\raggedright}p{#1}}
%\usepackage[style=numbering,natbib=true,backend=biber]{biblatex}
\usepackage{listings}
\lstset{ %
  language=R,                     % the language of the code
  basicstyle=\footnotesize\ttfamily,breaklines=true,
        % the size of the fonts that are used for the code
  numbers=left,                   % where to put the line-numbers
  numberstyle=\tiny\color{gray},  % the style that is used for the line-numbers
  stepnumber=1,                   % the step between two line-numbers. If it's 1, each line
                                  % will be numbered
  numbersep=5pt,                  % how far the line-numbers are from the code
  backgroundcolor=\color{white},  % choose the background color. You must add \usepackage{color}
  showspaces=false,               % show spaces adding particular underscores
  showstringspaces=false,         % underline spaces within strings
  showtabs=false,                 % show tabs within strings adding particular underscores
  frame=single,                   % adds a frame around the code
  rulecolor=\color{black},        % if not set, the frame-color may be changed on line-breaks within not-black text (e.g. commens (green here))
  tabsize=2,                      % sets default tabsize to 2 spaces
  captionpos=b,                   % sets the caption-position to bottom
  breaklines=true,                % sets automatic line breaking
  breakatwhitespace=false,        % sets if automatic breaks should only happen at whitespace
  title=\lstname,                 % show the filename of files included with \lstinputlisting;
                                  % also try caption instead of title
  keywordstyle=\color{blue},      % keyword style
  commentstyle=\color{gray},   % comment style
  stringstyle=\color{red},      % string literal style
  escapeinside={\%*}{*)},         % if you want to add a comment within your code
  morekeywords={*,...}            % if you want to add more keywords to the set
} 

\usepackage[inline]{enumitem}
\usepackage{tocloft}
%\renewcommand{\cftsecfont}{\small}
%\renewcommand{\cftsecpagefont}{\small}
\renewcommand{\cftsubsecfont}{\small}
\renewcommand{\cftsubsecpagefont}{\small}